\documentclass[oneside]{book}

\usepackage[utf8]{inputenc} 
\usepackage[T1]{fontenc}    
\usepackage[spanish]{babel}  
\usepackage{fancyhdr}         
\usepackage{amsmath, amssymb, amsfonts, amsthm}
\usepackage{hyperref}       
\usepackage{fullpage}       
\usepackage{titlesec}     
\usepackage{graphicx}          
\usepackage{ragged2e}       
\usepackage{multicol}
\usepackage{tcolorbox}
\usepackage{listings}   
\usepackage{tikz}
\usepackage{array}
\usepackage{booktabs}
\usepackage{amsmath}
\usepackage[backend=biber,style=numeric]{biblatex}
\usepackage{pgfplots}
\pgfplotsset{compat=1.18} 


\addbibresource{referencias.bib} 


\usetikzlibrary{shapes.geometric, arrows, positioning, calc, shapes.callouts}

\tikzstyle{startstop} = [rectangle, rounded corners, minimum width=3cm, minimum height=1cm,text centered, draw=black, fill=gray!30]
\tikzstyle{block} = [rectangle, draw, text centered, rounded corners, minimum height=1cm, minimum width=2.5cm]
\tikzstyle{data} = [cylinder, draw, shape aspect=0.3, shape border rotate=90, text centered, minimum height=2cm, minimum width=2cm, fill=blue!10]
\tikzstyle{process} = [rectangle, draw, text centered, rounded corners, minimum height=1cm, fill=blue!30]
\tikzstyle{arrow} = [thick,->,>=stealth]
\tikzstyle{output} = [ellipse, draw, text centered, minimum height=1cm, fill=blue!10]
\tikzstyle{decision} = [diamond, draw, text centered, minimum height=1cm, fill=red!20]
\tikzstyle{learning} = [rectangle, draw, text centered, rounded corners, minimum height=1cm, fill=red!30]



% Configuración de encabezados y pies de página
\setlength{\headheight}{15.0pt}   % Ajuste de headheight para evitar advertencias
\pagestyle{fancy}

\fancypagestyle{plain}{
    \fancyhf{}
    \fancyhead[R]{Tema \thechapter}
    \fancyhead[L]{Inteligencia Artificial}
    \fancyfoot[C]{\thepage}
    \renewcommand{\headrulewidth}{0.4pt}
}

% Redefinir el nombre de los capítulos de "Capítulo" a "Tema"
\renewcommand{\chaptername}{Tema}

% Personalizar el formato del capítulo para mostrar "Tema X: Título del Tema"
\titleformat{\chapter}[hang]
  {\bfseries\Huge}                   % Cambiado de \Huge a \LARGE para reducir tamaño
  {Tema \thechapter}                 % Formato del número del capítulo
  {1em}                               % Espacio entre el número y el título
  {}                                  % Formato del título del capítulo

% Ajustar el espaciado antes y después del título del capítulo
\titlespacing*{\chapter}{0pt}{-1em}{1em} % Reduce el espacio antes del título

% Eliminación de márgenes adicionales
\setlength{\parindent}{0pt}          % Sin indentación en párrafos

% Comando personalizado (si es necesario)
\newcommand{\Lagr}{\mathcal{L}}

\begin{document}

% Inclusión de la portada

\begin{titlepage}
\centering
{\bfseries\LARGE Universidad Europea de Madrid \par}
\vspace{0cm}
{\bfseries\LARGE Escuela de Arquitectura, Ingeniería y Diseño \par}
\vfill
\noindent\hrulefill \\
{\scshape\Huge Apuntes Inteligencia Artificial \par} %Título
\vspace{0.5cm}
{\scshape\Large Grado en Ingeniería Informática \par}
\vspace{0.5cm}
{\scshape\Large Inteligencia Artificil \par} %Semestre
\noindent\hrulefill \\
\vfill
{\bfseries\Large Profesor: Isabel Fuentes\par}
\vfill
{\Large Daniel Cuesta Sanz de Madrid}
\vfill
{\large\today \par} %esto crea la fecha de hoy

{\includegraphics[height=4cm]{logo.png}}
\end{titlepage}
\setcounter{page}{1}
\newpage

% Reiniciar el contador de páginas después de la portada
\setcounter{page}{1}

% Inclusión de Tema 1
\chapter{} % Define el título del capítulo
\textbf{El tema 1 es\dots}

\section{Introducción}

\subsection{cosas}


\newpage


\chapter{} 
\textbf{El tema 2 sirve como una introducción un poco mas detalladas a los sistemas inteligentes y los agentes inteligentes.}


\section{Agentes inteligentes}

Los agentes inteligentes son un \textit{subset} de los sistemas inteligentes que tienen la capacidad de percibir su entorno y actuar de manera autónoma para alcanzar sus objetivos. Estos agentes pueden ser simples o complejos, y su comportamiento puede ser predefinido o aprendido a través de la experiencia.
Estos agentes inteligentes no tienen porque ser físicos o virtuales, ya que pueden ser ambos. La arquitectura de un sistema inteligente normalmente tiene la siguiente estructura:

\begin{figure}[htbp]
    \centering
    
\begin{tikzpicture}[node distance=0.75cm]

% Nodes
\node (sensores) [block, fill=orange!70] {Sensores};
\node (receptor) [process, left=2cm of sensores] {Receptor};
\node (condicion) [process, fill=pink!50, below=1cm of receptor] {Condici\'on\ Acci\'on\ Reglas\ (If-then)};
\node (emisor) [process, below=1cm of condicion] {Emisor};
\node (actuadores) [block, fill=orange!70, below=3cm of sensores] {Actuadores};

% Arrows
\draw [arrow] (receptor) -- (condicion);
\draw [arrow] (sensores) -- (receptor);
\draw [arrow] (condicion) -- (emisor);
\draw [arrow] (emisor) -- (actuadores);

\end{tikzpicture}


    \caption{Arquitectura agentes inteligentes}
    \label{fig:agentes_inteligentes}
\end{figure}

Dicho de la manera más sencilla, un agente inteligente es: Un sistema informático que interpreta su \textbf{entorno} mediante \textbf{sensores} e interactúa con él usando \textbf{actuadores}. Este entonro como he dicho antes puede ser virtual o físico, ya sea un salón para una aspiradora robótica, un proyecto para el copilot\dots E igual con los sensores, de la misma manera que con los actuadores.

\subsection{Características de los agentes inteligentes}

Las características principales de los agentes inteligentes incluyen:

\begin{itemize}
    \item \textbf{Reactividad:} La capacidad de un agente para percibir su entorno y responder a los cambios en tiempo real.
    \item \textbf{Proactividad:} La capacidad de un agente para tomar la iniciativa y actuar de manera autónoma para alcanzar sus objetivos.
    \item \textbf{Adaptabilidad:} La capacidad de un agente para aprender de la experiencia y mejorar su comportamiento a lo largo del tiempo.
    \item \textbf{Comunicación:} La capacidad de un agente para interactuar y colaborar con otros agentes o sistemas para lograr objetivos comunes.
    \item \textbf{Autonomía:} La capacidad de un agente para operar sin intervención humana directa, tomando decisiones por sí mismo.
    \item \textbf{Racionalidad:} La capacidad de un agente para tomar decisiones basadas en la lógica y la información disponible para maximizar sus objetivos.
\end{itemize}

\subsection{Tipos de entornos}

Los entornos pueden ser de muchos tipos, y es además el aspecto más importante de un sistema inteligente, ya que este luego decidirá los sensores y actuadores que necesita, además de todas las acciones que puede hacer o hace. Los principales entornos que existen son:

\begin{itemize}
    \item \textbf{Entorno completamente observable vs. parcialmente observable:} En un entorno completamente observable, el agente tiene acceso a toda la información necesaria para tomar decisiones. En un entorno parcialmente observable, el agente solo tiene acceso a una parte de la información.
    \item \textbf{Entorno determinista vs. estocástico:} En un entorno determinista, las acciones del agente determinan de manera precisa el estado siguiente del entorno. En un entorno estocástico, hay un grado de incertidumbre en los resultados de las acciones del agente.
    \item \textbf{Entorno estático vs. dinámico:} En un entorno estático, el entorno no cambia mientras el agente está decidiendo su acción. En un entorno dinámico, el entorno puede cambiar mientras el agente está decidiendo su acción.
    \item \textbf{Entorno discreto vs. continuo:} En un entorno discreto, hay un número finito de estados y acciones posibles. En un entorno continuo, los estados y acciones pueden variar de manera continua.
    \item \textbf{Entorno episódico vs. secuencial:} En un entorno episódico, la experiencia del agente se divide en episodios independientes. En un entorno secuencial, las decisiones del agente afectan estados futuros y se deben considerar a largo plazo.
    \item \textbf{Entorno conocido vs. desconocido:} En un entorno conocido, el agente tiene conocimiento previo de las reglas del entorno. En un entorno desconocido, el agente debe aprender las reglas a través de la interacción.
\end{itemize}

\chapter{}
{\bfseries El tema 3 es una introducción a la minería de datos, especialmente técnicas como el web scrapping y además el aprendizaje 
automatico ligado a estas tecnicas.}

\section{\bfseries Introducción a la minería de datos y aprendizaje automático}

\subsection{\bfseries Minería de datos}
El proceso de minería de datos es un proceso iterativo que consiste en descubrir patrones en grandes volúmenes de datos, inicialmente
se encuentran procesos para preparar y limpiar los datos. El proceso de descubrimiento tiene las siguientes fases:
\begin{itemize}
    \item \textbf{Selección:} Se seleccionan los datos que se van a utilizar.
    \item \textbf{Procesamiento:} Se procesan y limpian para que puedan ser usados de una manera eficiente.
\end{itemize}

\subsection{\bfseries Extracción, transformación y carga de datos (ETL)}
Básicamente, el proceso ETL se encarga de extraer datos de diferentes fuentes, transformarlos y cargarlos en un almacén de datos.
Esta será una palabra que se escuchará mucho en el mundo de la minería de datos, ya que es un proceso fundamental para este.

\begin{tcolorbox}[title=Proceso de ETL]
\centering
\begin{tikzpicture}[node distance=0.5cm and 0.5cm, scale=0.8, every node/.style={scale=0.8}]

% Nodes for ETL process
\node[data] (data) {Datos};
\node[process, below right=0.3cm and 0.3cm of data] (selection) {Selecci\'on};
\node[data, below right=0.3cm and 0.3cm of selection] (selected) {Datos seleccionados};
\node[process, below right=0.3cm and 0.3cm of selected] (processing) {Procesamiento};
\node[data, below right=0.3cm and 0.3cm of processing] (processed) {Datos procesados};
\node[process, below right=0.3cm and 0.3cm of processed] (transformation) {Transformaci\'on};
\node[output, below right=0.3cm and 0.3cm of transformation] (transformed) {Datos transformados};


% Arrows for ETL
\draw[arrow] (data) -- (selection);
\draw[arrow] (selection) -- (selected);
\draw[arrow] (selected) -- (processing);
\draw[arrow] (processing) -- (processed);
\draw[arrow] (processed) -- (transformation);
\draw[arrow] (transformation) -- (transformed);


\end{tikzpicture}

\end{tcolorbox}

\subsection{\bfseries Aprendizaje automático}

El aprendizaje automático es una rama de la inteligencia artificial que permite a las máquinas aprender de los datos y hacer predicciones o tomar decisiones sin ser programadas explícitamente para ello. Existen varios tipos de aprendizaje automático:

\begin{itemize}
    \item \textbf{Aprendizaje supervisado:} Se entrena un modelo con datos etiquetados, es decir, datos que ya tienen una respuesta conocida.
    \item \textbf{Aprendizaje no supervisado:} Se entrena un modelo con datos no etiquetados y el objetivo es encontrar patrones o estructuras ocultas en los datos.
    \item \textbf{Aprendizaje semi-supervisado:} Combina una pequeña cantidad de datos etiquetados con una gran cantidad de datos no etiquetados durante el entrenamiento.
    \item \textbf{Aprendizaje por refuerzo:} Un agente aprende a tomar decisiones mediante la interacción con un entorno y la obtención de recompensas o castigos.
\end{itemize}


\section{\bfseries Extracción, transformación y carga de datos}

Antes de realizar web scrapping; \textbf{¿Es legal?}, sorprendentemente si, pero siguiendo diferentes reglas y normas:
\begin{multicols}{2}
    \begin{itemize}
        \item La información tiene que ser pública.
        \item Tienes que aceptar explícitamente los terminos de uso.
        \item Puedo hacer uso de los datos siempre que no lesione los derechos de autor.
    \end{itemize}
\end{multicols}

\subsection{\bfseries APIs}

Una API es un canal de comunicación, que permite que 2 software se comuniquen entre sí. Tiene 4 tipos diferentes de peticiones:
\begin{tcolorbox}[title=Peticiones API]
    \begin{itemize}
        \item \textbf{GET:} Se utiliza para obtener datos de un servidor.
        \item \textbf{POST:} Se utiliza para enviar datos a un servidor.
        \item \textbf{PUT:} Se utiliza para actualizar datos en un servidor.
        \item \textbf{DELETE:} Se utiliza para eliminar datos de un servidor.
    \end{itemize}
\end{tcolorbox}

Estas peticiones se realizan contra endpoints, que son formas de interactuar con este. También cabe destacar \textbf{REST}, que es un estilo de arquitectura de software que define un conjunto de restricciones para crear servicios web.

\textbf{\large Ejemplo:} 
\begin{lstlisting}[language=Python]
    import requests
    url = 'https://pokeapi.co/api/v2/pokemon/ditto'

    response = requests.get(url)
    print(response.json())
\end{lstlisting}

  En el ejemplo de encima hacemos un request a la API de Pokemon, y obtenemos la información del Pokemon Ditto. Dependiendo de la respuesta que nos encontremos, esta nos informa de si la petición ha sido correcta o no. Estas respuestas se dividen en diferentes códigos de estado:
  \begin{multicols}{2}
    \begin{itemize}
        \item 1xx: Información
        \item 2xx: Éxito
        \item 3xx: Redirección
        \item 4xx: Error del cliente
        \item 5xx: Error del servidor
    \end{itemize}
  \end{multicols}

Pero aqui nos encontramos un problema, ya que no es común que las APIs tengan una documentación clara y concisa, o directamente ni siquiera
ni nos permitan hacer uso de ellas. Para acceder a una API que no tiene una documentación clara, podemos hacer uso de la técnica de web scrapping o parsear el HTML directamente con BS4. \newline
\textbf{\large Ejemplo con BS4:} 
\begin{lstlisting}[language=Python]
    #Ejemplo con BeautifulSoup y la wikipedia de la UEM
    from bs4 import BeautifulSoup

    url = 'https://es.wikipedia.org/wiki/Universidad_Europea_de_Madrid'
    response = requests.get(url)
    soup = BeautifulSoup(response.content,"html.parser")
\end{lstlisting}

\section{\bfseries Representación y visualización de datos}
La representación de datos es una parte fundamental en la minería de datos, ya que nos permite visualizar los datos y darles un enfoque
diferente. Ya que aunque un analisis pueda tener datos muy similares, una vez visualizados estos pueden tener un enfoque diferente. Esto, por ejemplo
, se hace muy fácil de ver en el \textbf{Cuarteto de Anscombe}:

\begin{figure}[htbp]
    \centering
    \begin{minipage}{0.4\textwidth}
        \centering
        \begin{tikzpicture}
        \begin{axis}[
            title={Conjunto I},
            width=5cm, height=5cm, % Dimensiones más pequeñas
            xlabel={$x$}, ylabel={$y$},
            grid=major,
            xmin=0, xmax=20, ymin=2, ymax=14
        ]
        \addplot[only marks] coordinates {
            (10, 8.04) (8, 6.95) (13, 7.58) (9, 8.81) 
            (11, 8.33) (14, 9.96) (6, 7.24) (4, 4.26) 
            (12, 10.84) (7, 4.82) (5, 5.68)
        };
        \addplot[domain=0:20, red] {0.5 * x + 3};
        \end{axis}
        \end{tikzpicture}
    \end{minipage}
    \hfill
    \begin{minipage}{0.4\textwidth}
        \centering
        \begin{tikzpicture}
        \begin{axis}[
            title={Conjunto II},
            width=5cm, height=5cm, % Dimensiones más pequeñas
            xlabel={$x$}, ylabel={$y$},
            grid=major,
            xmin=0, xmax=20, ymin=2, ymax=14
        ]
        \addplot[only marks] coordinates {
            (10, 9.14) (8, 8.14) (13, 8.74) (9, 8.77) 
            (11, 9.26) (14, 8.10) (6, 6.13) (4, 3.10) 
            (12, 9.13) (7, 7.26) (5, 4.74)
        };
        \addplot[domain=0:20, red] {0.5 * x + 3};
        \end{axis}
        \end{tikzpicture}
    \end{minipage}

    \vspace{0.5cm}

    \begin{minipage}{0.4\textwidth}
        \centering
        \begin{tikzpicture}
        \begin{axis}[
            title={Conjunto III},
            width=5cm, height=5cm, % Dimensiones más pequeñas
            xlabel={$x$}, ylabel={$y$},
            grid=major,
            xmin=0, xmax=20, ymin=2, ymax=14
        ]
        \addplot[only marks] coordinates {
            (10, 7.46) (8, 6.77) (13, 12.74) (9, 7.11) 
            (11, 7.81) (14, 8.84) (6, 6.08) (4, 5.39) 
            (12, 8.15) (7, 6.42) (5, 5.73)
        };
        \addplot[domain=0:20, red] {0.5 * x + 3};
        \end{axis}
        \end{tikzpicture}
    \end{minipage}
    \hfill
    \begin{minipage}{0.4\textwidth}
        \centering
        \begin{tikzpicture}
        \begin{axis}[
            title={Conjunto IV},
            width=5cm, height=5cm, % Dimensiones más pequeñas
            xlabel={$x$}, ylabel={$y$},
            grid=major,
            xmin=0, xmax=20, ymin=2, ymax=14
        ]
        \addplot[only marks] coordinates {
            (8, 6.58) (8, 5.76) (8, 7.71) (8, 8.84) 
            (8, 8.47) (8, 7.04) (8, 5.25) (19, 12.50) 
            (8, 5.56) (8, 7.91) (8, 6.89)
        };
        \addplot[domain=0:20, red] {0.5 * x + 3};
        \end{axis}
        \end{tikzpicture}
    \end{minipage}
\end{figure}

Todos estos gráficos tienen datos muy similares (media, varianza, correlación\ldots) pero sus representaciones gráficas son totalmente diferentes.
Esto nos demuestra la importancia de la representación de datos para poder comprender una función o un conjunto de datos. 

\chapter{}
{\bfseries El tema 4 por fin introduce la inteligencia artificial, comenzando por el aprendizaje supervisado}

\section{Introducción al machine learning} 
El machine learning, o aprendizaje automático, es una rama de la inteligencia artificial que se centra en el desarrollo de algoritmos y técnicas que permiten a las computadoras aprender de y hacer predicciones sobre datos. A través del uso de modelos matemáticos y estadísticos, las máquinas pueden identificar patrones y tomar decisiones con mínima intervención humana.

Todo el proceso tiene \textbf{5 pasos principales}:
\begin{enumerate}
    \item Recoger Datos.
    \item Preprocesamiento de datos
    \item Elegir el mejor modelo.
    \item Entrenar y ajustar los parámetros del modelo.
    \item Ver si los resultados del modelo predicen correctamente nuevos datos.
\end{enumerate}

\subsection{Recoger Datos}

Los datos se pueden obtener de muchas maneras: Mediante web scraping, desde una API o base de datos\dots
Pero como vamos a realizar aprendizaje supervisado necesitamos \textbf{datos etiquetados}.

\renewcommand{\arraystretch}{1.5} % Espaciado entre filas
\newcolumntype{C}[1]{>{\centering\arraybackslash}m{#1}} % Columnas centradas

\begin{table}[h!]
\centering
\setlength{\arrayrulewidth}{0.3mm} % Grosor de las líneas de la tabla
\setlength{\tabcolsep}{5pt} % Espaciado horizontal en celdas

\begin{tabular}{|C{1cm}|C{1cm}|C{1cm}|C{1cm}|C{1cm}|C{1cm}|C{1cm}|C{1cm}|C{1cm}|C{1cm}|C{1cm}|}
\hline
\textbf{age} & \textbf{gender} & \textbf{height} & \textbf{weight} & \textbf{ap\_hi} & \textbf{ap\_lo} & \textbf{cholest} & \textbf{gluc} & \textbf{smoke} & \textbf{alco} & \textbf{cardio} \\
\hline
46.0 & F & 172 & 112 & 120 & 80 & 1 & 1 & 0 & 0 & YES \\
\hline
44.0 & M & 170 & 69 & 120 & 70 & 1 & 1 & 0 & 1 & NO \\
\hline
45.5 & M & 159 & 49 & 120 & 70 & 1 & 1 & 0 & 0 & NO \\
\hline
39.7 & F & 164 & 48 & 110 & 70 & 1 & 2 & 1 & 1 & YES \\
\hline
63.3 & F & 180 & 104 & 120 & 85 & 2 & 2 & 0 & 0 & NO \\
\hline
\end{tabular}
\caption{Ejemplo de datos etiquetados}
\label{tab:datos_etiquetados}
\end{table}

Por ejemplo en la tabla \ref{tab:datos_etiquetados} tenemos datos de pacientes y si han tenido un ataque al corazón o no. Los datos etiquetado se elegirán dependiendo de la variable que queramos predecir:

\[ \{ (x[i], y[i]) \mid x[i] \in \mathbb{R}^n,\ y[i] \in \{c_1, c_2, \dots, c_m\},\ i = 1, \dots, N \} \]

Todo esto con el objetivo de generar un modelo acorde a la fórmula: $ \hat{Y} = f\left(x, w\right) $ Siendo $x_n$ cada variable y $w$ los pesos de cada variable.

\subsection{Preprocesamiento de datos}

El preprocesamiento de datos es una etapa crítica en el proceso de machine learning, ya que los datos sin procesar pueden contener errores, valores atípicos o información redundante que puede afectar la precisión de los modelos. Algunas técnicas comunes de preprocesamiento de datos incluyen la limpieza de datos, la normalización de datos y la selección de características. A continuación se detallarán las técnicas de preprocesamiento a usar.

\subsubsection{Missing Values}
Primero se mira si hay \textit{missing values}. Dependiendo del tamaño del dataset, se puede eliminar las muestras del dataset (si es muy grande) o intentar predecir los con otros modelos para datasets pequeños.

\subsubsection{Visualizar para ver si hay outliers}
Los \textit{outliers} son  valores atípicos que pueden afectar negativamente el rendimiento de los modelos de machine learning. Se pueden visualizar mediante gráficos de caja y bigotes, histogramas o diagramas de dispersión.
Las maneras más tipicas de visualizar para ver si hay outliers es:

% Primera gráfica: Visualización de los datos
\begin{figure}[h!]
    \centering
    \begin{tikzpicture}[scale=0.9]
        % Ejes
        \draw[arrow] (0, 0) -- (5, 0) node[right] {X};
        \draw[arrow] (0, 0) -- (0, 3) node[above] {Y};

        % Puntos
        \foreach \x/\y in {0.5/0.25, 1/0.5, 1.5/0.75, 2/1, 2.5/1.5, 3/2} {
            \node[circle, fill=black, scale=0.5] at (\x, \y) {};
        }
        \foreach \x/\y in {0.5/2} {
            \node[circle, fill=red, scale=0.5] at (\x, \y) {};
        }
    \end{tikzpicture}
    \caption{Visualizaci\'on de los datos}
\end{figure}

% Segunda gráfica: Box-plot
\begin{figure}[h!]
    \centering
    \begin{tikzpicture}[scale=0.9]
        % Ejes
        \draw[arrow] (0, 0) -- (4, 0) node[right] {Category};
        \draw[arrow] (0, 0) -- (0, 3) node[above] {Age};

        % Caja izquierda
        \draw[thick, red] (0.5, 0.5) rectangle (1.5, 2);
        \draw[thick] (1, 0.25) -- (1, 2.5); % Línea central

        % Caja derecha
        \draw[thick, green] (2.5, 1) rectangle (3.5, 2.5);
        \draw[thick] (3, 0.75) -- (3, 3); % Línea central

        % Outliers
        \node[circle, fill=red, scale=0.5] at (1.05, 2.75) {};
        \node[circle, fill=red, scale=0.5] at (1.1, 2.8) {};
        \node[circle, fill=red, scale=0.5] at (0.95, 2.65) {};
        \node[circle, fill=red, scale=0.5] at (1, 2.6) {}; % Outlier izquierda superior
    \end{tikzpicture}
    \caption{Box-plot}
\end{figure}


Frente a los outliers de las figuras (destacados en rojo), ¿Qué se puede hacer? Si son varios hay que verificar si hay datos mal recogidos, o que se itnerpretan mal en la base de datos, con pocos outliers puede ser:
\begin{itemize}
    \item Casos con comportamiento distinto pero con razón: \textit{\textbf{True Informative Outliers}}
    \item Casualidad: \textbf{\textit{False informative Outliers}}
    \item Que no tengamos suficientes datos y sean normales.
\end{itemize}

\subsubsection{Codificar la variables categóricas}
Las variables categóricas son aquellas que representan una categoría o clase en lugar de un valor numérico. Para poder utilizar estas variables en los modelos de machine learning, es necesario codificarlas en un formato numérico. Algunas técnicas comunes de codificación de variables categóricas incluyen la codificación one-hot y la codificación ordinal.
El problema que tenemos con esto es el \textit{skewnewss} o asimetría, que es una medida de la simetría de la distribución de los datos. Una distribución puede ser simétrica, sesgada a la derecha (positiva) o sesgada a la izquierda (negativa). A continuación se muestran ejemplos de distribuciones con diferentes tipos de asimetría.

\begin{figure}[h!]
\centering
\begin{tikzpicture}
\begin{axis}[
    width=0.45\textwidth,
    height=0.3\textwidth,
    title={Distribución con skewness negativa},
    xlabel={Valores},
    ylabel={Frecuencia},
    yticklabels={,,},
    xticklabels={,,},
    domain=-3:3,
    samples=100,
    smooth,
    no markers
]
\addplot[fill=blue!20, draw=blue] {exp(-0.5*(x+1)^2)};
\end{axis}
\end{tikzpicture}
\hfill
\begin{tikzpicture}
\begin{axis}[
    width=0.45\textwidth,
    height=0.3\textwidth,
    title={Distribución con skewness positiva},
    xlabel={Valores},
    ylabel={Frecuencia},
    yticklabels={,,},
    xticklabels={,,},
    domain=-3:3,
    samples=100,
    smooth,
    no markers
]
\addplot[fill=red!20, draw=red] {exp(-0.5*(x-1)^2)};
\end{axis}
\end{tikzpicture}
\caption{Ejemplos de distribuciones con skewness negativa (izquierda) y positiva (derecha)}
\label{fig:skewness}
\end{figure}


%bibliografia
\newpage
{$\space$\par}
\rhead[Referencias]{Referencias}

\defbibheading{myheading}[Referencias]{%
\section{Refencias}
}
\printbibliography[heading=myheading,title={Referencias}]

\end{document}
