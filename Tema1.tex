\textbf{El tema 1 es una introducción a la inteligencia artificial así como una breve historia de la misma. 
Se explican los conceptos básicos y se introducen los sistemas inteligentes.}

\section{¿Qué es la inteligencia artificial?}
La inteligencia artificial no tiene una definición exacta como tal, pero se puede entender como la capacidad de una máquina para imitar las funciones cognitivas humanas, tales como el aprendizaje y la resolución de problemas.
\subsection{Historia de la IA}
La inteligencia artificial ha tenido (por ahora) una historia muy corta, se podría dividir su vida hasta hoy en 5 etapas principales:
\begin{itemize}
    \item \textbf{1943-1956 Genesis de la IA:} Surge el término 'Inteligencia Artificial' y los primeros computadores y juegos de ajedrez.
    \item \textbf{1952-1969 Entusiasmo Inicial:} Numerosos proyecto surgen, y se crean nuevas herramientas.
    \item \textbf{1966-1973 Etapa de crisis:} Decepción ante los resultados en los últimos años, se redimensionan los problemas.
    \item \textbf{1969-1979 Resurgimiento:} Se cambia el enfoque por completo, ya no se intentan resolver problemas generales, sino que solo los de dominios concretos.
    \item \textbf{1980-2024 Industria Actual:} La última frontera de la IA, se invierte más dinero que nunca en la tencología.
\end{itemize}

\section{¿Qué es un sistema inteligente?}

Un sistema inteligente es aquel que puede percibir su entorno y tomar decisiones para maximizar sus posibilidades de éxito en alguna tarea o conjunto de tareas. Estos sistemas pueden ser diseñados para aprender y adaptarse a nuevas situaciones, mejorando su rendimiento con el tiempo. Los sistemas inteligentes pueden ser clasificados en varias categorías, tales como:

\begin{itemize}
    \item \textbf{Sistemas basados en conocimiento:} Utilizan un conjunto de reglas predefinidas para tomar decisiones. Son fáciles de entender y diseñar, pero tienen limitaciones en cuanto a su capacidad de adaptación.
    \item \textbf{Sistemas basados en aprendizaje:} Utilizan algoritmos de aprendizaje automático para mejorar su rendimiento a partir de datos. Estos sistemas pueden adaptarse a nuevas situaciones y mejorar con el tiempo.
    \item \textbf{Sistemas híbridos:} Combinan características de los sistemas basados en reglas y los sistemas basados en aprendizaje para aprovechar las ventajas de ambos enfoques.
\end{itemize}

Un ejemplo de sistema inteligente es un asistente virtual, como Siri o Alexa, que puede entender comandos de voz, buscar información en internet, y realizar tareas como enviar mensajes o configurar recordatorios. Estos asistentes utilizan técnicas de procesamiento de lenguaje natural y aprendizaje automático para mejorar su capacidad de entender y responder a las solicitudes de los usuarios.

\subsection{Sistema basado en el conocimiento}
Un sistema basado en el conocimiento presenta una arquitectura con tres componentes principales que interactuan entre ellos para conseguir un objetivo:
\begin{enumerate}
    \item \textbf{Motor de inferencia:} Se encarga del control del problema buscando en la base de conocimiento.
    \item \textbf{Base de Conocimiento:} Almaneca las reglas de conocimiento
    \item \textbf{Base de hechos:} Es la que contiene los datos iniciales del problema
\end{enumerate}

En base a estos tres componentes, el motor de inferencia y la base de conocimiento va creando nuevos hechos que luego se almacenan en la base de hechos.

\section{Aplicaciones de la IA}

La IA tiene muchas aplicaciones posibles, algunas de las cuales se detallan en la Tabla \ref{tab:aplicaciones_ia}. Estas aplicaciones abarcan desde el control avanzado de sistemas hasta la minería de datos, pasando por el diagnóstico y reparación, los agentes inteligentes y el asesoramiento. Cada una de estas categorías tiene casos de uso específicos que demuestran el potencial de la inteligencia artificial para transformar diversas industrias y mejorar la eficiencia y efectividad de numerosos procesos.

\begin{table}[h!]
\centering
\begin{tabular}{|l|l|}
\hline
\textbf{Categoría} & \textbf{Casos de uso} \\ \hline
Control avanzado de sistemas & \begin{tabular}[c]{@{}l@{}}Unidades de transporte ferroviario\\ Planificación de procesos complejos\\ Transbordador espacial\end{tabular} \\ \hline
Diagnóstico y reparación & \begin{tabular}[c]{@{}l@{}}Medicina\\ Locomotoras eléctricas\\ Detección de yacimientos\end{tabular} \\ \hline
Agentes Inteligentes & \begin{tabular}[c]{@{}l@{}}Comercio electrónico automatizado\\ Videojuegos\end{tabular} \\ \hline
Asesoramiento & \begin{tabular}[c]{@{}l@{}}Entidades de crédito\\ Detección de fraudes\end{tabular} \\ \hline
Minería de datos & \begin{tabular}[c]{@{}l@{}}Modelado financiero\\ Astronomía\\ Biología\end{tabular} \\ \hline
\end{tabular}
\caption{Aplicaciones de la IA}
\label{tab:aplicaciones_ia}
\end{table}





